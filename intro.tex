\section {Introduction}\label{sec:intro}
\subsection{Purpose}
This document defines the mission, goals and objectives, organization and responsibilities of
 \VRO Data Management Operations.

\subsection{Mission Statement}
\quote{  Maintain, improve and operate a suite of Vera C. Rubin data management services to produce
and serve to the community high-quality data products from the Legacy Survey of Space and Time.
}


\subsection{Goals and Objectives}
These are similar to our construction goals outlined in \citeds{LDM-294}.
Rubin Data Management Operations will:
\begin{itemize}
\item Produce the data products as outlined in \citeds{LSE-61}
\item Maintain and improve data production mechanisms.
\item Maintain and improve data access mechanisms.
\item Maintain and improve data abstraction mechanisms.
\item Assess current and operations-era technologies for use in providing engineered solutions for \VRO.
\item Maintain appropriate cybersecurity measures throughout \VRO and especially on external facing services.
\item Document the operational procedures associated with using and maintaining DM capabilities.
\item Evaluate, select, recruit, hire/contract and direct permanent staff, contract, and in-kind resources in Rubin and from partner organizations participating in DM initiatives.

\end{itemize}


The goals in selecting and, where necessary, developing Rubin software solutions are:

\begin{itemize}
\item   We prefer to acquire and configure existing, off-the-shelf, solutions. Where no satisfactory off-the-shelf solutions are available, we develop the software and hardware systems necessary to meet our objectives.
This extends into maintenance where we will continue to probe choices and may replace custom systems with off-the-shelf solutions where appropriate.
\item The software architecture is actively managed at the subsystem level. A well engineered and cleanly designed codebase is less buggy, more maintainable, and makes developers who work on it more productive. We continue to follow and maintain the developer guide\footnote{\url{developer.lsst.io}}.
\item  Other than when prohibited by licensing, security, or other similar considerations, all newly developed source code, and in particular that pertaining to scientific algorithms, is public.
Our primary goals in publicizing the code are to simplify reproducibility of LSST data products and to provide insight into algorithms used.
Achieving these goals requires that the software must be properly documented.
\item Background decision material on choices made will be documented in technical notes with "DMTN", "RTN" or similar series handles. (see \url{lsst.io})
\end{itemize}
