\documentclass[OPS,lsstdraft,authoryear,toc]{lsstdoc}
\input{meta}

% Package imports go here.

% Local commands go here.

%If you want glossaries
%\input{aglossary.tex}
%\makeglossaries

\title{Management and Execution plan for Data Management Operations.}

% Optional subtitle
% \setDocSubtitle{A subtitle}

\author{%
William O'Mullane
}

\setDocRef{RTN-046}
\setDocUpstreamLocation{\url{https://github.com/lsst/rtn-046}}

\date{\vcsDate}

% Optional: name of the document's curator
% \setDocCurator{The Curator of this Document}

\setDocAbstract{%
This is the management plan for operations of Data Management - this includes software products and data products.
}

% Change history defined here.
% Order: oldest first.
% Fields: VERSION, DATE, DESCRIPTION, OWNER NAME.
% See LPM-51 for version number policy.
\setDocChangeRecord{%
  \addtohist{0.1}{2023-01-10}{Draft.}{William O'Mullane}
}


\begin{document}

% Create the title page.
\maketitle
% Frequently for a technote we do not want a title page  uncomment this to remove the title page and changelog.
% use \mkshorttitle to remove the extra pages

\section {Introduction}\label{sec:intro}
\subsection{Purpose}
This document defines the mission, goals and objectives, organization and responsibilities of
 \VRO Data Management Operations.

\subsection{Mission Statement}
\quote{  Maintain, improve and operate a suite of Vera C. Rubin data management services to produce
and serve to the community high-quality data products from the Legacy Surevey of Space and Time.
}


\subsection{Goals and Objectives}
Rubin Data Management Operations will:
\begin{itemize}
\item Produce the data products, data access mechanisms, r LSST (with approval by others).
\item Assess current and operations-era technologies for use in providing engineered solutions to the requirements.
\item Define a secure computing, communications, and storage infrastructure and services architecture underlying DM.
\item Select, implement, construct, test, document, and deploy the data management infrastructure, middleware, applications, and external interfaces.
\item Adopt appropriate cybersecurity measures throughout the DM subsystem and especially on external facing services.
\item Document the operational procedures associated with using and maintaining DM capabilities.
\item Evaluate, select, recruit, hire/contract and direct permanent staff, contract, and in-kind resources in LSST and from partner organizations participating in LSST DM initiatives.

\end{itemize}



\section{Functionality based teams and organisation} \label{sec:org}

\begin{figure}
\begin{centering}
\includegraphics[width=0.9\textwidth]{images/DpOrg}
	\caption{Reporting lines in Data Management Operations.
\label{fig:org}}
\end{centering}
\end{figure}


While \figref{fig:org} Shows the reporting structure \figref{fig:arcorg} puts this more in a operations concept. We consider the main functions to be Data Production and Data Serving, these are supported by the data abstraction team and the data facilities.

\begin{figure}
\begin{centering}
\includegraphics[width=1.2\textwidth]{images/dm_ops_org_arch.pdf}
\caption{Functions in operations of Rubin Data Management.\label{fig:arcorg}}
\end{centering}
% https://docs.google.com/presentation/d/1s8j3I8ByB7NojvhT79_dbh8ZdOBFnPBodS0vNXc3SRU/edit#slide=id.p
\end{figure}

\subsection{ Data services} \label{sec:servingdata}

All services associated with data serving are in this group.
As depicted in \figref{fig:archorg} this includes:

\begin{itemize}
\item The Science Platform
\item Developer Services Infrastructure
\item Qserv advanced Database
\end{itemize}

A more complete list of items under may be found in the \secref{sec:products}

\subsection{Making Data} \label{sec:makingdata}
text here

\subsection{Data Architecture and service software} \label{sec:dataservices}
text here

\subsection{Infrastructure} \label{sec:infrastructure}
text here

\section{Chile DevOps} \label{sec:devops}
text here

\subsection{USDF}
\subsection{FRDF}
\subsection{UKDF}
\subsection{Cloud DF}

\section{Products}

\section{Roles}



\appendix
% Include all the relevant bib files.
% https://lsst-texmf.lsst.io/lsstdoc.html#bibliographies
\section{References} \label{sec:bib}
\renewcommand{\refname}{} % Suppress default Bibliography section
\bibliography{local,lsst,lsst-dm,refs_ads,refs,books}

% Make sure lsst-texmf/bin/generateAcronyms.py is in your path
\section{Acronyms} \label{sec:acronyms}
\addtocounter{table}{-1}
\begin{longtable}{p{0.145\textwidth}p{0.8\textwidth}}\hline
\textbf{Acronym} & \textbf{Description}  \\\hline

API & Application Programming Interface \\\hline
CC & Change Control \\\hline
CDF & Cumulative Distribution Function \\\hline
CM & Configuration Management \\\hline
DF & Data Facility \\\hline
DM & Data Management \\\hline
DMTN & DM Technical Note \\\hline
EFD & Engineering and Facility Database \\\hline
FITS & Flexible Image Transport System \\\hline
FTE & Full-Time Equivalent \\\hline
LDM & LSST Data Management (Document Handle) \\\hline
LSE & LSST Systems Engineering (Document Handle) \\\hline
LSST & Legacy Survey of Space and Time (formerly Large Synoptic Survey Telescope) \\\hline
MW & Milky Way \\\hline
OCPS & OCS Controlled Pipeline System \\\hline
OPS & Operations \\\hline
PanDA &  Production ANd Distributed Analysis system \\\hline
QSERV & LSST Query Services \\\hline
RSP & Rubin Science Platform \\\hline
RTN & Rubin Technical Note \\\hline
UKDF & United Kingdom Data Facility \\\hline
US & United States \\\hline
USDF & United States Data Facility \\\hline
VO & Virtual Observatory \\\hline
VRO & (not to be used)Vera C. Rubin Observatory \\\hline
\end{longtable}

% If you want glossary uncomment below -- comment out the two lines above
%\printglossaries





\end{document}
